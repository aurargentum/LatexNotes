\documentclass[12pt]{article}
\usepackage{amsmath}
\usepackage{amssymb}
\usepackage{tikz}
\usepackage{babel}
\usepackage{titling}
\setlength{\droptitle}{-4em}     % Eliminate the default vertical space
\addtolength{\droptitle}{-24pt}   % Only a guess. Use this for adjustment
\title{Thermodynamics}
\date{}
	\begin{document}
	\maketitle
	\begin{itemize}
		\item For conservative forces $\nabla\times F=0$
		\section{ First Law of Thermodynamics}
		\item $dU + dE_k + dE_p = dQ +dW$
			\begin{itemize}
			\item $dU$ is generally in per unit mass.
			\item $dE_k+dE_p$ is generally \textbf{negligible}.
			\item $dQ$ is the heat \textbf{given} to the system.
			\item $dW$ is the work done \textbf{on} the system.
			\item $dW=-P\Delta V$
			\end{itemize}
		\item \textbf{\underline{Energy Balance}}\\
		 \[\frac{d(mU)_{cv}}{dt} + \Delta \cdot [ \{ (U + PV)+ \frac{1}{2}\upsilon^2 + gz\}\dot{m}]=\dot{Q} + \dot{W}\]
			\begin{itemize}
				\item	$\frac{d(mU)_{cv}}{dt}$ here, \textbf{cv} means \textbf{control volume} which becomes \textit{zero} in \textit{steady state} \& \textit{slow process}.
				\item $U+PV =H$
				\item $\dot{W}$ is the \textbf{shaft work}
			\end{itemize}
			\item \textbf{\underline{Flow Calorimeter}}\\
			Used to find enthalpy of a system\\
			\textbf{\[H=Q\]}\\
			

\tikzset{every picture/.style={line width=0.75pt}} %set default line width to 0.75pt        

\begin{tikzpicture}[x=0.75pt,y=0.75pt,yscale=-1,xscale=1]
%uncomment if require: \path (0,300); %set diagram left start at 0, and has height of 300

%Straight Lines [id:da4467873958256816] 
\draw    (220.5,129) -- (220.5,237) ;
%Straight Lines [id:da5871924515081283] 
\draw    (101.5,237) -- (220.5,237) ;
%Straight Lines [id:da9973433774650685] 
\draw    (100,128) -- (101.5,237) ;
%Curve Lines [id:da21154801854876026] 
\draw    (150.5,187) .. controls (309.5,166) and (45.5,183) .. (179.5,145) ;
%Curve Lines [id:da1699153982177869] 
\draw    (149.5,249) .. controls (243.5,224) and (71.5,213) .. (150.5,187) ;
%Straight Lines [id:da558837079497294] 
\draw    (261.5,145) -- (179.5,145) ;
%Shape: Rectangle [id:dp29835155943009484] 
\draw   (261.5,126) -- (331.5,126) -- (331.5,166) -- (261.5,166) -- cycle ;
%Straight Lines [id:da7723720923046552] 
\draw    (261.5,145) -- (332.5,145.97) ;
\draw [shift={(334.5,146)}, rotate = 180.78] [color={rgb, 255:red, 0; green, 0; blue, 0 }  ][line width=0.75]    (10.93,-3.29) .. controls (6.95,-1.4) and (3.31,-0.3) .. (0,0) .. controls (3.31,0.3) and (6.95,1.4) .. (10.93,3.29)   ;
%Shape: Rectangle [id:dp7175199509703801] 
\draw   (362.5,126) -- (432.5,126) -- (432.5,166) -- (362.5,166) -- cycle ;
%Straight Lines [id:da20309345036114157] 
\draw    (334.5,146) -- (434.5,146.98) ;
\draw [shift={(436.5,147)}, rotate = 180.56] [color={rgb, 255:red, 0; green, 0; blue, 0 }  ][line width=0.75]    (10.93,-3.29) .. controls (6.95,-1.4) and (3.31,-0.3) .. (0,0) .. controls (3.31,0.3) and (6.95,1.4) .. (10.93,3.29)   ;
%Straight Lines [id:da25255582998008186] 
\draw    (302.5,182) -- (301.62,168) ;
\draw [shift={(301.5,166)}, rotate = 446.42] [color={rgb, 255:red, 0; green, 0; blue, 0 }  ][line width=0.75]    (10.93,-3.29) .. controls (6.95,-1.4) and (3.31,-0.3) .. (0,0) .. controls (3.31,0.3) and (6.95,1.4) .. (10.93,3.29)   ;

% Text Node
\draw (273,105) node [anchor=north west][inner sep=0.75pt]   [align=left] {Heater};
% Text Node
\draw (369,106) node [anchor=north west][inner sep=0.75pt]   [align=left] {T2 \ \ \ \ P2};
% Text Node
\draw (304.5,185) node [anchor=north west][inner sep=0.75pt]   [align=left] {Q};


\end{tikzpicture}
			\item \textbf{\underline{Thermal Equilibrium}}
			\begin{itemize}
			\item Mechanical (Sum of all forces is 0)
			\item Thermal (No gradient of temp) ($\nabla T =0$)
			\item Chemical (No gradient of chemical potential)
			\end{itemize}
			\item \textbf{\underline{ Quasi-Static Process}}
			\begin{itemize}
				\item $PdV$ \& $TdS$ like terms can be applied only in a quasi-static process
				\item $\tau_r$ or the \textit{relaxation time} is the time required to achieve equilibrium when disturbed
			\end{itemize}
			\item \textbf{\underline{Reversible Processes}}
			\begin{itemize}
				\item Frictionless
				\item we can write $-PdV$ for only reversible processes
			\end{itemize}
			\item Specific heat $(C) \equiv \frac{\partial Q}{\partial T} \Rightarrow$ So, $mC\partial T = Q$\\
			$\displaystyle C_V \equiv \left( \frac{\partial U}{\partial T}\right)_V \&\ C_P \equiv \left( \frac{\partial H}{\partial T}\right)_P$\\
			$C_P - C_V = R$
			\item \textbf{\underline{ Gibbs Phase Rule}}
			\[ F = 2- \pi +N\]
			\begin{itemize}
				\item $F$ is the no. of \textit{intensive} degree of freedom
				\item $\pi$ is the no. of phases
				\item $N$ is the no. of chemical species
			\end{itemize}
			\item \textbf{\underline{Vol. properties of pure substance}}\\
			

\tikzset{every picture/.style={line width=0.75pt}} %set default line width to 0.75pt        

\begin{tikzpicture}[x=0.75pt,y=0.75pt,yscale=-1,xscale=1]
%uncomment if require: \path (0,300); %set diagram left start at 0, and has height of 300

%Shape: Axis 2D [id:dp33843863525946605] 
\draw  (44.5,270.6) -- (365.5,270.6)(76.6,6) -- (76.6,300) (358.5,265.6) -- (365.5,270.6) -- (358.5,275.6) (71.6,13) -- (76.6,6) -- (81.6,13)  ;
%Curve Lines [id:da4451632950459832] 
\draw    (96.5,221) .. controls (144.02,210.11) and (226.82,111.99) .. (240.12,69.28) ;
\draw [shift={(240.5,68)}, rotate = 465.95] [color={rgb, 255:red, 0; green, 0; blue, 0 }  ][line width=0.75]    (10.93,-3.29) .. controls (6.95,-1.4) and (3.31,-0.3) .. (0,0) .. controls (3.31,0.3) and (6.95,1.4) .. (10.93,3.29)   ;
%Curve Lines [id:da5106172941765282] 
\draw    (183,155) .. controls (223.5,169) and (276.5,148) .. (316.5,118) ;
\draw   (105.61,222.59) -- (96.18,220.72) -- (103.03,213.97) ;
%Straight Lines [id:da1638478712960444] 
\draw  [dash pattern={on 0.84pt off 2.51pt}]  (77.5,119) -- (357.5,119) ;
%Straight Lines [id:da5234502090106499] 
\draw  [dash pattern={on 0.84pt off 2.51pt}]  (315.5,268) -- (316.5,45) ;

% Text Node
\draw (125,75) node [anchor=north west][inner sep=0.75pt]   [align=left] {Solid};
% Text Node
\draw (336,75) node [anchor=north west][inner sep=0.75pt]   [align=left] {Super Critical Fluids};
% Text Node
\draw (202.68,84.3) node [anchor=north west][inner sep=0.75pt]  [rotate=-310.47] [align=left] {Melting};
% Text Node
\draw (104.57,196.05) node [anchor=north west][inner sep=0.75pt]  [rotate=-318.98,xslant=0.05] [align=left] {Sublimation};
% Text Node
\draw (253,85) node [anchor=north west][inner sep=0.75pt]   [align=left] {Liquid};
% Text Node
\draw (219,203) node [anchor=north west][inner sep=0.75pt]   [align=left] {Vapours};
% Text Node
\draw (340,176) node [anchor=north west][inner sep=0.75pt]   [align=left] {Gas};
% Text Node
\draw (336,99) node [anchor=north west][inner sep=0.75pt]   [align=left] {Density like liquid \&\\Viscosity like vapour};

\draw   (208.65,114.59) -- (216.34,122.28)(216.34,114.59) -- (208.65,122.28) ;
\draw   (312.65,114.15) -- (320.35,121.85)(320.35,114.15) -- (312.65,121.85) ;
\end{tikzpicture}\\
This curve is a slice of constant $V$.\\


\tikzset{every picture/.style={line width=0.75pt}} %set default line width to 0.75pt        

\begin{tikzpicture}[x=0.75pt,y=0.75pt,yscale=-1,xscale=1]
%uncomment if require: \path (0,300); %set diagram left start at 0, and has height of 300

%Shape: Axis 2D [id:dp33843863525946605] 
\draw  (44.5,270.6) -- (365.5,270.6)(76.6,6) -- (76.6,300) (358.5,265.6) -- (365.5,270.6) -- (358.5,275.6) (71.6,13) -- (76.6,6) -- (81.6,13)  ;
%Curve Lines [id:da6433099767212852] 
\draw    (134.5,237.91) .. controls (139.29,210.09) and (152.58,168.32) .. (171.96,133.36) .. controls (188.99,102.65) and (210.71,77.2) .. (235.5,71.09) ;
%Curve Lines [id:da28122346552298627] 
\draw    (235.5,71.09) .. controls (292.5,69) and (307.5,256.01) .. (362.5,251) ;
%Straight Lines [id:da34415215866171356] 
\draw    (78,238) -- (368.5,238) ;
%Curve Lines [id:da26463197271954053] 
\draw    (117.5,24) .. controls (123.5,80) and (110.5,208) .. (134.5,237.91) ;
%Curve Lines [id:da7227455695285275] 
\draw    (109.5,25) .. controls (115.5,81) and (91.5,237) .. (76.6,270.6) ;
%Straight Lines [id:da10008258811346382] 
\draw  [dash pattern={on 0.84pt off 2.51pt}]  (75.5,71) -- (235.5,71.09) ;
%Straight Lines [id:da7918039013302539] 
\draw  [dash pattern={on 0.84pt off 2.51pt}]  (236.5,239) -- (235.5,71.09) ;
%Curve Lines [id:da584739321813705] 
\draw    (165.5,37) .. controls (151.71,76.4) and (144.71,97.37) .. (177,116.14) ;
\draw [shift={(178.5,117)}, rotate = 209.2] [color={rgb, 255:red, 0; green, 0; blue, 0 }  ][line width=0.75]    (10.93,-3.29) .. controls (6.95,-1.4) and (3.31,-0.3) .. (0,0) .. controls (3.31,0.3) and (6.95,1.4) .. (10.93,3.29)   ;

% Text Node
\draw (207,274) node [anchor=north west][inner sep=0.75pt]   [align=left] {V};
% Text Node
\draw (55,133) node [anchor=north west][inner sep=0.75pt]   [align=left] {P};
% Text Node
\draw (98,210) node [anchor=north west][inner sep=0.75pt]   [align=left] {s+l};
% Text Node
\draw (253,136) node [anchor=north west][inner sep=0.75pt]   [align=left] {l+g};
% Text Node
\draw (204,242) node [anchor=north west][inner sep=0.75pt]   [align=left] {s+g};
% Text Node
\draw (324,150) node [anchor=north west][inner sep=0.75pt]   [align=left] {gas};
% Text Node
\draw (146,18) node [anchor=north west][inner sep=0.75pt]   [align=left] {sat. liq. vap. line};

\draw   (208.65,114.59) -- (216.34,122.28)(216.34,114.59) -- (208.65,122.28) ;
\draw   (312.65,114.15) -- (320.35,121.85)(320.35,114.15) -- (312.65,121.85) ;
\end{tikzpicture}\\
Super cooled liquid exists at temp lower than the boiling point at that pressure.\\
	\item \textbf{\underline{ Volume Expansivity}}\\
	Think of it as fractional increase in volume per unit rise in temperature (division by $V$ because \textit{fractional}.
	\[ \beta = \frac{1}{V}\left(\frac{\partial V}{\partial T}\right)_P\]
	\item \textbf{\underline{Isothermal Compressibility}}

\begin{gather*}
\kappa \equiv \frac{-1}{V}\left(\frac{\partial V}{\partial T}\right)_{P}\\
\frac{\partial V}{V} =\beta \partial T-\kappa \partial P\\
\Longrightarrow ln\left(\frac{V_{2}}{V_{1}}\right) =\beta ( T_{2} -T_{1}) -\kappa ( P_{2} -P_{1})
\end{gather*}
Fractional decrease in volume per unit change in pressure. Here, negative sign $-$ is to maintain sign convention.\\
For incompressible fluids $\beta = \kappa =0$
	\item \textbf{\underline{Equation of State}}\\
	

\tikzset{every picture/.style={line width=0.75pt}} %set default line width to 0.75pt        

\begin{tikzpicture}[x=0.75pt,y=0.75pt,yscale=-1,xscale=1]
%uncomment if require: \path (0,300); %set diagram left start at 0, and has height of 300

%Shape: Axis 2D [id:dp2807289791745755] 
\draw  (25.5,274.81) -- (484.5,274.81)(71.4,4.81) -- (71.4,304.81) (477.5,269.81) -- (484.5,274.81) -- (477.5,279.81) (66.4,11.81) -- (71.4,4.81) -- (76.4,11.81)  ;
%Straight Lines [id:da2086769047106125] 
\draw    (71.5,132.81) -- (290.5,132.81) ;
%Straight Lines [id:da6646271482375132] 
\draw    (71.5,132.81) -- (260.5,52.81) ;
%Straight Lines [id:da5273095834234696] 
\draw    (71.5,132.81) -- (235.5,209.81) ;
%Straight Lines [id:da4767599796440273] 
\draw    (71.5,132.81) -- (276.5,87.81) ;
%Shape: Brace [id:dp5612025296007479] 
\draw   (323.5,225.81) .. controls (328.17,225.86) and (330.53,223.56) .. (330.58,218.89) -- (331.39,144.89) .. controls (331.46,138.22) and (333.83,134.92) .. (338.5,134.97) .. controls (333.83,134.92) and (331.54,131.56) .. (331.61,124.89)(331.58,127.89) -- (332.42,50.89) .. controls (332.47,46.22) and (330.17,43.86) .. (325.5,43.81) ;

% Text Node
\draw (289,78) node [anchor=north west][inner sep=0.75pt]   [align=left] {N{\tiny 2}};
% Text Node
\draw (296,122) node [anchor=north west][inner sep=0.75pt]   [align=left] {Ar};
% Text Node
\draw (238,203) node [anchor=north west][inner sep=0.75pt]   [align=left] {O{\tiny 2}};
% Text Node
\draw (268,41) node [anchor=north west][inner sep=0.75pt]   [align=left] {H{\tiny 2}};
% Text Node
\draw (344,124) node [anchor=north west][inner sep=0.75pt]   [align=left] {at fixed temp};
% Text Node
\draw (33,104) node [anchor=north west][inner sep=0.75pt]   [align=left] {PV};
% Text Node
\draw (251,280) node [anchor=north west][inner sep=0.75pt]   [align=left] {V};
% Text Node
\draw (444,35) node [anchor=north west][inner sep=0.75pt]   [align=left] {P is inversely proportional\\to V only at high volume \ \\\& low pressure};


\end{tikzpicture}\\
		\begin{itemize}
		\item \[T (Kelvin\ scale) = \frac{273.16 (PV)^*}{227118.8\ cm^3\ bar\ mol^{-1}}\]
		\item $PV= a(T)+b(T)P+c(T)P^2$\\
		$a,b,c$ are temperature dependent constants.
		\item $\displaystyle Z=\frac{PV}{RT}\\
		Z= 1+B'P+C'P^2+...$
		\end{itemize}
		\item \textbf{\underline{Virial Equation of State}}\\
		\begin{align}
		 \displaystyle PV &= a+bP+cP^2+...\nonumber \\ 
		 &= a[1+B'P+C'P+...]\nonumber \\
		Z &= 1+\frac{B}{V}+\frac{C}{V^2}+\frac{D}{V^3}+...\nonumber \\
		B' &=\frac{B}{RT}\ \ ,\ C'=\frac{C-B^2}{(RT)^2} \nonumber
		\end{align}\\
		if we take first two terms,\\
		$\displaystyle Z= 1+\frac{BP}{RT}=\frac{PV}{RT}\\\\
		\implies P= \frac{RT}{V-B}$---(1)\\\\
		$\displaystyle Z= 1+ \frac{B}{V} = \frac{PV}{RT}\\\\
		P=\frac{RT}{V}\left(1+\frac{B}{V}\right)---(2)$\\
		(2) is a special case of (1) when $B<<V$.\\
		During Ideality (low $P$)\\
		$PV^{ig}=a=RT$
		\item \textbf{\underline{Vanderwaal Eqn. of State}}\\
		$\displaystyle \left(P+\frac{a}{V^2}\right)\left(V-b\right)=RT$\\
		where,\\
		$\frac{a}{V^2} = interaction force\\
		b= finite\ size\ of\ molecules$\\
		$\displaystyle a=\frac{27}{64}\frac{R^2T_c^2}{P_c} \ , \
		 b=\frac{1}{8}\frac{RT_c}{P_c}\ , \
		 P_c=\frac{a}{27b^2}\ , \
		 T_c=\frac{8a}{27Rb}\  ,\
		 V_c=3b\\
		 Z_c=\frac{P_cV_c}{RT_c}=\frac{3}{8}\\
		 P_r=\frac{P}{P_c}\ \ , \ V_r=\frac{V}{V_c}\ \ , \ \ T_r=\frac{T}{T_c}$\\
		 Fluids with any of the above 2 same have same compressibility factor \& deviate from ideal gas behaviours by same factor.\\
		 

\tikzset{every picture/.style={line width=0.75pt}} %set default line width to 0.75pt        

\begin{tikzpicture}[x=0.75pt,y=0.75pt,yscale=-1,xscale=1]
%uncomment if require: \path (0,333); %set diagram left start at 0, and has height of 333

%Shape: Axis 2D [id:dp14561986409004402] 
\draw  (57.5,293.71) -- (439.5,293.71)(95.7,22.81) -- (95.7,323.81) (432.5,288.71) -- (439.5,293.71) -- (432.5,298.71) (90.7,29.81) -- (95.7,22.81) -- (100.7,29.81)  ;
%Curve Lines [id:da42338337171872364] 
\draw    (96,183) .. controls (152.5,204.81) and (261.5,186.81) .. (294.5,143.81) ;
%Curve Lines [id:da9949803942359926] 
\draw    (96,183) .. controls (160.5,178.81) and (255.5,129.81) .. (286.5,95.81) ;
%Curve Lines [id:da9275747719071034] 
\draw    (96,183) .. controls (178.5,282.81) and (253.5,271.81) .. (302.5,200.81) ;

% Text Node
\draw (373,121) node [anchor=north west][inner sep=0.75pt]  [font=\Large]  {$P^{sat} \propto e^{T} \ ;\ \frac{d\left[ logP^{sat}_{r}\right]}{d\left(\frac{1}{T}\right)} \approx a$};
% Text Node
\draw (557,82) node [anchor=north west][inner sep=0.75pt]  [font=\scriptsize] [align=left] {constant for many \\simple gases};
% Text Node
\draw (275,63) node [anchor=north west][inner sep=0.75pt]  [font=\footnotesize]  {$T_{r} =2.00$};
% Text Node
\draw (289,129) node [anchor=north west][inner sep=0.75pt]  [font=\footnotesize]  {$T_{r} =1.5$};
% Text Node
\draw (305,185) node [anchor=north west][inner sep=0.75pt]  [font=\footnotesize]  {$T_{r} =1.2$};
% Text Node
\draw (70,126) node [anchor=north west][inner sep=0.75pt]    {$Z$};
% Text Node
\draw (265,301) node [anchor=north west][inner sep=0.75pt]    {$P_{r}$};


\end{tikzpicture}\\
	\item \textbf{\underline{Acentric Flow}}\ : non sphericity of molecules\\
	\\
	According to law of corresponding state for 3 parameters, if $P^r\ ,\ T^r\ ,\ \& \ \omega$ of the species is same, then the other reduced quantities will be the same.

\tikzset{every picture/.style={line width=0.75pt}} %set default line width to 0.75pt        

\begin{tikzpicture}[x=0.75pt,y=0.75pt,yscale=-1,xscale=1]
%uncomment if require: \path (0,300); %set diagram left start at 0, and has height of 300

%Shape: Axis 2D [id:dp7307728274378104] 
\draw  (44.5,44.88) -- (354.5,44.88)(75.5,232.81) -- (75.5,24) (347.5,49.88) -- (354.5,44.88) -- (347.5,39.88) (70.5,225.81) -- (75.5,232.81) -- (80.5,225.81)  ;
%Straight Lines [id:da9730901653341038] 
\draw    (75.5,44.88) -- (278.5,106.81) ;
%Straight Lines [id:da07506785541601912] 
\draw    (75.5,44.88) -- (290.5,181.81) ;
%Straight Lines [id:da5255702125706964] 
\draw  [dash pattern={on 0.84pt off 2.51pt}]  (230.5,43.81) -- (229.5,143.81) ;
%Straight Lines [id:da8449412583070479] 
\draw  [dash pattern={on 0.84pt off 2.51pt}]  (230,93.81) -- (76.5,92.81) ;
%Shape: Brace [id:dp18500985869791953] 
\draw   (228.5,142.81) .. controls (233.16,143.09) and (235.63,140.9) .. (235.91,136.24) -- (236.39,128.21) .. controls (236.79,121.56) and (239.32,118.37) .. (243.97,118.65) .. controls (239.32,118.37) and (237.19,114.9) .. (237.58,108.25)(237.4,111.24) -- (238.07,100.22) .. controls (238.34,95.56) and (236.15,93.09) .. (231.5,92.81) ;

% Text Node
\draw (217,23) node [anchor=north west][inner sep=0.75pt]   [align=left] {1.4};
% Text Node
\draw (52,85) node [anchor=north west][inner sep=0.75pt]   [align=left] {\mbox{-}1};
% Text Node
\draw (369,24) node [anchor=north west][inner sep=0.75pt]    {$\frac{1}{Tr}$};
% Text Node
\draw (6.59,143) node [anchor=north west][inner sep=0.75pt]  [xslant=-0.03]  {$log\left( P^{sat}_{r}\right)$};
% Text Node
\draw (249,114) node [anchor=north west][inner sep=0.75pt]  [font=\scriptsize] [align=left] {acentric factor $\displaystyle \omega $};
% Text Node
\draw (372,159) node [anchor=north west][inner sep=0.75pt]    {$\omega =-1-log\left( P^{sat}_{r}\right)$};


\end{tikzpicture}\\



\tikzset{every picture/.style={line width=0.75pt}} %set default line width to 0.75pt        

\begin{tikzpicture}[x=0.75pt,y=0.75pt,yscale=-1,xscale=1]
%uncomment if require: \path (0,300); %set diagram left start at 0, and has height of 300

%Shape: Axis 2D [id:dp9531828459927774] 
\draw  (54.5,280.81) -- (409.5,280.81)(90,10.81) -- (90,310.81) (402.5,275.81) -- (409.5,280.81) -- (402.5,285.81) (85,17.81) -- (90,10.81) -- (95,17.81)  ;
%Curve Lines [id:da9518057067169535] 
\draw    (89.5,252.81) .. controls (248.5,248.81) and (156.5,149.81) .. (262.5,149.81) ;
%Curve Lines [id:da9335134325216686] 
\draw    (262.5,149.81) .. controls (385.5,150.81) and (283.5,55.81) .. (384.5,55.81) ;
%Straight Lines [id:da18011856925836078] 
\draw  [dash pattern={on 0.84pt off 2.51pt}]  (90.5,151.81) -- (262.5,149.81) ;
%Straight Lines [id:da6365527368707031] 
\draw  [dash pattern={on 0.84pt off 2.51pt}]  (89.5,56.81) -- (384.5,55.81) ;
%Straight Lines [id:da5814496183892577] 
\draw  [dash pattern={on 0.84pt off 2.51pt}]  (383.5,279.81) -- (384.5,55.81) ;
%Straight Lines [id:da3935959235017661] 
\draw  [dash pattern={on 0.84pt off 2.51pt}]  (261.5,281.81) -- (262.5,149.81) ;

% Text Node
\draw (56,43) node [anchor=north west][inner sep=0.75pt]    {$\frac{7}{2} R$};
% Text Node
\draw (61,136) node [anchor=north west][inner sep=0.75pt]    {$\frac{5}{2} R$};
% Text Node
\draw (63,239) node [anchor=north west][inner sep=0.75pt]    {$\frac{3}{2} R$};
% Text Node
\draw (42,166) node [anchor=north west][inner sep=0.75pt]  [font=\scriptsize] [align=left] {Diatomic};
% Text Node
\draw (26,267) node [anchor=north west][inner sep=0.75pt]  [font=\scriptsize] [align=left] {Monoatomic};
% Text Node
\draw (358,282) node [anchor=north west][inner sep=0.75pt]   [align=left] {1000K};
% Text Node
\draw (246,283) node [anchor=north west][inner sep=0.75pt]   [align=left] {100K};
% Text Node
\draw (113,282) node [anchor=north west][inner sep=0.75pt]   [align=left] {10K};
% Text Node
\draw (282,155) node [anchor=north west][inner sep=0.75pt]  [font=\scriptsize] [align=left] {Rotational Degree \\of Freedom};
% Text Node
\draw (420,36) node [anchor=north west][inner sep=0.75pt]   [align=left] {For Hydrogen};
% Text Node
\draw (57,6) node [anchor=north west][inner sep=0.75pt]    {$C_{V}$};
% Text Node
\draw (421,271) node [anchor=north west][inner sep=0.75pt]    {$T$};


\end{tikzpicture}\\
	\item \textbf{\underline{Sensible Heat Effects}}\\
	$\displaystyle \partial U = \left(\frac{\partial U}{\partial T}\right)_V\partial T+\left(\frac{\partial V}{\partial V}\right)_T \partial V$ [Pure Substances in Single Phase]\\
	Here :\ $\displaystyle \partial U = \left(\frac{\partial U}{\partial T}\right)_V\partial T=C_VdT$ \\
	$\displaystyle \partial H = \left(\frac{\partial H}{\partial T}\right)_P\partial T+\left(\frac{\partial H}{\partial P}\right)_T \partial P$\\
	Here: \ $\displaystyle \partial H = \left(\frac{\partial H}{\partial T}\right)_P\partial T=C_PdT$ \\
	For incompressible substance:\ $Q=\Delta V = \int C_V dT$\\
	For ideal gas/ low pressure gas:\ $Q=\Delta H = \int C_PdT$\\
	\begin{itemize}
		\item \textbf{\underline{Empirical Relation}}\\
		\\
		$\displaystyle \frac{C_P^{ig}}{R}=A+BT+CT^2+DT^{-2}\\ \\
		\frac{C_V^{ig}}{R}=\frac{C_P^{ig}}{R}-1$
	\end{itemize}
	$\displaystyle \int_{T_0}^TC_PdT=R\left[A(T_0-1)T_0+\frac{B}{2}T_0^2(\tau^2-1)+\frac{C}{3}T_0^3(\tau^3-1)+\frac{D}{T_0}\left(\frac{\tau-1}{\tau}\right)\right]$\\
	$\displaystyle \tau=\frac{T}{T_0} \ \ \ \ \ \left<\frac{C_P}{R}\right>(T-T_0)=|_{T_0}^{T}\frac{C_P}{R}dT$\\
	\[\displaystyle \left< \frac{C_P}{R}\right> = A+\frac{B_0T_0}{2}(Z+1)+\frac{C}{3}T_0^3(Z^2+Z+1)+\frac{D}{T_0}\left(\frac{1}{Z}\right) \label{eqn:avgcp} \tag{1}\]\\
	To find $\tau$ for a given $\Delta H$ use this
	\begin{itemize}
		\item Guess a $\tau$ \& substitute in \eqref{eqn:avgcp}
		\item Substitute $<C_P>$ in below equation to find next $\tau$
		\[\tau = 1+\frac{\Delta H}{T_0<C_P>}\]
		\item Repeat the process till $ [ \tau^{(N)}-\tau^{(N-1)}] < 10^{-5}$
	\end{itemize}
	\item \textbf{\underline{Latent Heat of Pure Substance}}
	

\tikzset{every picture/.style={line width=0.75pt}} %set default line width to 0.75pt        

\begin{tikzpicture}[x=0.75pt,y=0.75pt,yscale=-1,xscale=1]
%uncomment if require: \path (0,562); %set diagram left start at 0, and has height of 562

%Shape: Axis 2D [id:dp029718662297509635] 
\draw  (43,233.73) -- (450.5,233.73)(83.75,26) -- (83.75,256.81) (443.5,228.73) -- (450.5,233.73) -- (443.5,238.73) (78.75,33) -- (83.75,26) -- (88.75,33)  ;
%Straight Lines [id:da5876507747884343] 
\draw    (84,212) -- (176.5,184.81) ;
%Straight Lines [id:da848347882070567] 
\draw    (176.5,160.81) -- (266.5,143.81) ;
%Straight Lines [id:da5755645762576527] 
\draw    (266.5,119.81) -- (353.5,100.81) ;
%Straight Lines [id:da32630927891906414] 
\draw    (176.5,160.81) -- (176.5,184.81) ;
%Straight Lines [id:da27007005516274274] 
\draw    (266.5,119.81) -- (266.5,143.81) ;
%Straight Lines [id:da5730995917090059] 
\draw  [dash pattern={on 0.84pt off 2.51pt}]  (266.5,143.81) -- (265.5,234.81) ;
%Straight Lines [id:da8651600642552204] 
\draw  [dash pattern={on 0.84pt off 2.51pt}]  (176.5,184.81) -- (175.5,235.81) ;

%Shape: Axis 2D [id:dp05088279692074826] 
\draw  (43,517.73) -- (450.5,517.73)(83.75,310) -- (83.75,540.81) (443.5,512.73) -- (450.5,517.73) -- (443.5,522.73) (78.75,317) -- (83.75,310) -- (88.75,317)  ;
%Curve Lines [id:da07362566517423641] 
\draw    (114.5,366.81) .. controls (128.5,383.81) and (153.5,395.81) .. (176.5,401.81) ;
%Curve Lines [id:da6499150014361094] 
\draw    (176.5,401.81) .. controls (204.5,435.81) and (226.5,446.81) .. (266.5,449.81) ;
%Curve Lines [id:da08694672769626632] 
\draw    (266.5,449.81) .. controls (278.5,466.81) and (302.5,489.81) .. (403.5,491.81) ;
%Straight Lines [id:da6478634875178761] 
\draw    (185.5,390.81) -- (384.61,322.46) ;
\draw [shift={(386.5,321.81)}, rotate = 521.05] [color={rgb, 255:red, 0; green, 0; blue, 0 }  ][line width=0.75]    (10.93,-3.29) .. controls (6.95,-1.4) and (3.31,-0.3) .. (0,0) .. controls (3.31,0.3) and (6.95,1.4) .. (10.93,3.29)   ;
%Straight Lines [id:da6214893012978076] 
\draw    (270.5,441.81) -- (385.11,323.25) ;
\draw [shift={(386.5,321.81)}, rotate = 494.03] [color={rgb, 255:red, 0; green, 0; blue, 0 }  ][line width=0.75]    (10.93,-3.29) .. controls (6.95,-1.4) and (3.31,-0.3) .. (0,0) .. controls (3.31,0.3) and (6.95,1.4) .. (10.93,3.29)   ;
%Shape: Brace [id:dp9498483470236354] 
\draw   (268.5,143.81) .. controls (271.38,143.81) and (272.82,142.37) .. (272.82,139.49) -- (272.82,139.49) .. controls (272.82,135.37) and (274.26,133.31) .. (277.15,133.31) .. controls (274.26,133.31) and (272.82,131.25) .. (272.82,127.14)(272.82,128.99) -- (272.82,127.14) .. controls (272.82,124.25) and (271.38,122.81) .. (268.5,122.81) ;

% Text Node
\draw (55,122.81) node [anchor=north west][inner sep=0.75pt]    {$H$};
% Text Node
\draw (441,246.81) node [anchor=north west][inner sep=0.75pt]    {$T$};
% Text Node
\draw (436,526.81) node [anchor=north west][inner sep=0.75pt]    {$T$};
% Text Node
\draw (256,237.81) node [anchor=north west][inner sep=0.75pt]    {$T_{b}$};
% Text Node
\draw (168,237.81) node [anchor=north west][inner sep=0.75pt]    {$T_{m}$};
% Text Node
\draw (56,411.81) node [anchor=north west][inner sep=0.75pt]    {$G$};
% Text Node
\draw (127,393.81) node [anchor=north west][inner sep=0.75pt]    {$S$};
% Text Node
\draw (128,163.81) node [anchor=north west][inner sep=0.75pt]    {$S$};
% Text Node
\draw (214,441.81) node [anchor=north west][inner sep=0.75pt]    {$L$};
% Text Node
\draw (215,128.81) node [anchor=north west][inner sep=0.75pt]    {$L$};
% Text Node
\draw (314,488.81) node [anchor=north west][inner sep=0.75pt]    {$G$};
% Text Node
\draw (306,81.81) node [anchor=north west][inner sep=0.75pt]    {$G$};
% Text Node
\draw (388,304) node [anchor=north west][inner sep=0.75pt]   [align=left] {Non differentiability here is due\\ to jumps in $\displaystyle H\ vs\ T$ diagram};
% Text Node
\draw (457,55) node [anchor=north west][inner sep=0.75pt]   [align=left] {Almost same for all \\substances};
% Text Node
\draw (279,123) node [anchor=north west][inner sep=0.75pt]   [align=left] {$\displaystyle \Delta H_{vap}$};


\end{tikzpicture}
	\item \textbf{\underline{ Clayperon Equation}}
	\[ \left(\frac{\partial P}{\partial T}\right)_{Coexistence\ Curve} = \frac{\Delta H_{Transition}}{T\Delta V_{Transition}}\]
	

\tikzset{every picture/.style={line width=0.75pt}} %set default line width to 0.75pt        

\begin{tikzpicture}[x=0.75pt,y=0.75pt,yscale=-1,xscale=1]
%uncomment if require: \path (0,300); %set diagram left start at 0, and has height of 300

%Shape: Axis 2D [id:dp7977284228043576] 
\draw  (27,262.33) -- (304.5,262.33)(54.75,24) -- (54.75,288.81) (297.5,257.33) -- (304.5,262.33) -- (297.5,267.33) (49.75,31) -- (54.75,24) -- (59.75,31)  ;
%Shape: Axis 2D [id:dp7811381287214613] 
\draw  (346,262.52) -- (623.5,262.52)(373.75,24.19) -- (373.75,289) (616.5,257.52) -- (623.5,262.52) -- (616.5,267.52) (368.75,31.19) -- (373.75,24.19) -- (378.75,31.19)  ;
%Curve Lines [id:da1415095459704263] 
\draw    (137.5,164.81) .. controls (183.5,154.81) and (195.5,144.81) .. (214.5,109.81) ;
%Curve Lines [id:da5465405867991592] 
\draw    (475.5,166.81) .. controls (521.5,156.81) and (533.5,146.81) .. (552.5,111.81) ;
%Straight Lines [id:da8122087067281746] 
\draw    (137.5,164.81) -- (92.51,87.54) ;
\draw [shift={(91.5,85.81)}, rotate = 419.78999999999996] [color={rgb, 255:red, 0; green, 0; blue, 0 }  ][line width=0.75]    (10.93,-3.29) .. controls (6.95,-1.4) and (3.31,-0.3) .. (0,0) .. controls (3.31,0.3) and (6.95,1.4) .. (10.93,3.29)   ;
%Straight Lines [id:da2952231486229828] 
\draw    (475.5,166.81) -- (500.86,91.71) ;
\draw [shift={(501.5,89.81)}, rotate = 468.66] [color={rgb, 255:red, 0; green, 0; blue, 0 }  ][line width=0.75]    (10.93,-3.29) .. controls (6.95,-1.4) and (3.31,-0.3) .. (0,0) .. controls (3.31,0.3) and (6.95,1.4) .. (10.93,3.29)   ;
%Curve Lines [id:da2702221694796706] 
\draw    (137.5,164.81) .. controls (130.64,185.39) and (99.77,210.77) .. (73.13,220.25) ;
\draw [shift={(71.5,220.81)}, rotate = 341.57] [color={rgb, 255:red, 0; green, 0; blue, 0 }  ][line width=0.75]    (10.93,-3.29) .. controls (6.95,-1.4) and (3.31,-0.3) .. (0,0) .. controls (3.31,0.3) and (6.95,1.4) .. (10.93,3.29)   ;
%Curve Lines [id:da3092448614966199] 
\draw    (475.5,166.81) .. controls (468.64,187.39) and (437.77,212.77) .. (411.13,222.25) ;
\draw [shift={(409.5,222.81)}, rotate = 341.57] [color={rgb, 255:red, 0; green, 0; blue, 0 }  ][line width=0.75]    (10.93,-3.29) .. controls (6.95,-1.4) and (3.31,-0.3) .. (0,0) .. controls (3.31,0.3) and (6.95,1.4) .. (10.93,3.29)   ;

% Text Node
\draw (158,44) node [anchor=north west][inner sep=0.75pt]   [align=left] {For Water};
% Text Node
\draw (467,42) node [anchor=north west][inner sep=0.75pt]   [align=left] {For other gases};
% Text Node
\draw (31,122) node [anchor=north west][inner sep=0.75pt]    {$P$};
% Text Node
\draw (349,128) node [anchor=north west][inner sep=0.75pt]    {$P$};
% Text Node
\draw (169,271) node [anchor=north west][inner sep=0.75pt]    {$T$};
% Text Node
\draw (490,267) node [anchor=north west][inner sep=0.75pt]    {$T$};


\end{tikzpicture}
	For water \\
	$\displaystyle \left( \frac{\partial P}{\partial T}\right)_{S\to L}<0$ as $\Delta H_{S\to L} >0$ and $V_L-V_S <0$\\
	If we are given $T_0$ we can find $\displaystyle \left(\frac{\partial P}{\partial T}\right)$ by above curve, $\Delta H_{Transition}$ by $\Delta H\ vs\ T$ curve and $\Delta V_{Transition}$ by $P\ vs\ V$ graph.
	\item \textbf{\underline{ Trouton Rule}}
	\begin{itemize}
	\item \underline{Normal BP} : In this, the temperature of boiling point at a pressure of $1\ atm$
	\item $\displaystyle \frac{\Delta H_{vap}}{RT_n} \simeq 10$ for almost every substance
	\item $\displaystyle \frac{\Delta H_n}{RT_n}= \frac{1.092(lnP_C-1.013)}{0.93-T_n}$; It is basically $\displaystyle ln\left(\frac{P_C}{P}\right)$ where $P$ is $1\ atm$.
	\end {itemize}
	\item \textbf{\underline{ Watson's Equation}}
	$\displaystyle \frac{\Delta H_2}{\Delta H_1} = \left(\frac{1-T_{r2}}{1-T_{r1}}\right)^{0.38}$\\
	\\\\\\\\\\
	
	\section{ Second Law of Thermodynamics}
	\item No apparatus can operate in such a way that it's only effect is to convert heat observed by the system completely to work done by a system.
	\item No process is possible which consists solely in the transfer of heat from one temperature level to a higher temperature level.
	\item \underline{ Heat Engines}: Cycles  which involve absorption of heat at a higher temperature and rejection of heat at lower temperature.
	\[ Q_H-Q_C=W\]
	\begin{itemize}
	\item $Q_H$ is heat absorbed
	\item $Q_C$ is the heat rejected, it can never be zero
	\end{itemize}
	\item \underline{ Reversible Engine}
	

\tikzset{every picture/.style={line width=0.75pt}} %set default line width to 0.75pt        

\begin{tikzpicture}[x=0.75pt,y=0.75pt,yscale=-1,xscale=1]
%uncomment if require: \path (0,300); %set diagram left start at 0, and has height of 300

%Shape: Rectangle [id:dp5631656843184523] 
\draw   (135,21) -- (205,21) -- (205,61) -- (135,61) -- cycle ;
%Shape: Rectangle [id:dp503605071585652] 
\draw   (135,224) -- (205,224) -- (205,264) -- (135,264) -- cycle ;
%Shape: Circle [id:dp5895564099477936] 
\draw   (145,142) .. controls (145,128.19) and (156.19,117) .. (170,117) .. controls (183.81,117) and (195,128.19) .. (195,142) .. controls (195,155.81) and (183.81,167) .. (170,167) .. controls (156.19,167) and (145,155.81) .. (145,142) -- cycle ;
%Straight Lines [id:da5297330048770588] 
\draw    (169.5,223.81) -- (169.98,169) ;
\draw [shift={(170,167)}, rotate = 450.5] [color={rgb, 255:red, 0; green, 0; blue, 0 }  ][line width=0.75]    (10.93,-3.29) .. controls (6.95,-1.4) and (3.31,-0.3) .. (0,0) .. controls (3.31,0.3) and (6.95,1.4) .. (10.93,3.29)   ;
%Straight Lines [id:da4089792757501536] 
\draw    (170,117) -- (170.48,62.19) ;
\draw [shift={(170.5,60.19)}, rotate = 450.5] [color={rgb, 255:red, 0; green, 0; blue, 0 }  ][line width=0.75]    (10.93,-3.29) .. controls (6.95,-1.4) and (3.31,-0.3) .. (0,0) .. controls (3.31,0.3) and (6.95,1.4) .. (10.93,3.29)   ;
%Curve Lines [id:da10826564520941317] 
\draw    (172.76,159.31) .. controls (195.54,153.45) and (192.16,130.09) .. (175.12,123.39) ;
\draw [shift={(173.5,122.81)}, rotate = 377.53] [color={rgb, 255:red, 0; green, 0; blue, 0 }  ][line width=0.75]    (10.93,-3.29) .. controls (6.95,-1.4) and (3.31,-0.3) .. (0,0) .. controls (3.31,0.3) and (6.95,1.4) .. (10.93,3.29)   ;
\draw [shift={(170.5,159.81)}, rotate = 349.11] [color={rgb, 255:red, 0; green, 0; blue, 0 }  ][line width=0.75]    (10.93,-3.29) .. controls (6.95,-1.4) and (3.31,-0.3) .. (0,0) .. controls (3.31,0.3) and (6.95,1.4) .. (10.93,3.29)   ;

% Text Node
\draw (162,31) node [anchor=north west][inner sep=0.75pt]    {$T_{C}$};
% Text Node
\draw (160,234) node [anchor=north west][inner sep=0.75pt]    {$T_{H}$};
% Text Node
\draw (172,170) node [anchor=north west][inner sep=0.75pt]    {$Q_{H}$};
% Text Node
\draw (197,134) node [anchor=north west][inner sep=0.75pt]    {$W$};
% Text Node
\draw (44,132) node [anchor=north west][inner sep=0.75pt]    {$dU=0$};
% Text Node
\draw (401,88) node [anchor=north west][inner sep=0.75pt]    {$dU=Q_{H} -Q_{C} -W$};
% Text Node
\draw    (402,158) -- (563,158) -- (563,203) -- (402,203) -- cycle  ;
\draw (405,162) node [anchor=north west][inner sep=0.75pt]    {$\eta =\frac{W}{|Q_{H} |} \ =\ 1-\left| \frac{Q_{C}}{Q_{H}}\right| $};


\end{tikzpicture}
	\item \underline{Carnot Engine}
	\begin{itemize}
	\item Carnot's Theorem: For two given heat reserviors, no heat engine can have a thermal efficiency higher than a Carnot engine.
	\item Corollary: All Carnot engines operating between the same temperature have the same $\eta$
	\end{itemize}
	

\tikzset{every picture/.style={line width=0.75pt}} %set default line width to 0.75pt        

\begin{tikzpicture}[x=0.75pt,y=0.75pt,yscale=-1,xscale=1]
%uncomment if require: \path (0,300); %set diagram left start at 0, and has height of 300

%Shape: Axis 2D [id:dp9758893239756717] 
\draw  (51.5,260.21) -- (422.5,260.21)(88.6,29.81) -- (88.6,285.81) (415.5,255.21) -- (422.5,260.21) -- (415.5,265.21) (83.6,36.81) -- (88.6,29.81) -- (93.6,36.81)  ;
%Curve Lines [id:da8985523211546317] 
\draw    (156,84) .. controls (177.5,107.81) and (267.5,138.81) .. (331.5,139.81) ;
\draw [shift={(239.97,124.24)}, rotate = 196.82999999999998] [color={rgb, 255:red, 0; green, 0; blue, 0 }  ][line width=0.75]    (10.93,-4.9) .. controls (6.95,-2.3) and (3.31,-0.67) .. (0,0) .. controls (3.31,0.67) and (6.95,2.3) .. (10.93,4.9)   ;
%Curve Lines [id:da4683870633971089] 
\draw    (178,152) .. controls (199.5,175.81) and (289.5,206.81) .. (353.5,207.81) ;
\draw [shift={(261.97,192.24)}, rotate = 16.83] [color={rgb, 255:red, 0; green, 0; blue, 0 }  ][line width=0.75]    (10.93,-3.29) .. controls (6.95,-1.4) and (3.31,-0.3) .. (0,0) .. controls (3.31,0.3) and (6.95,1.4) .. (10.93,3.29)   ;
%Curve Lines [id:da7023418866775458] 
\draw    (178,152) .. controls (168.5,140.81) and (150.5,94.81) .. (156,84) ;
\draw [shift={(161.83,119.14)}, rotate = 429.44] [color={rgb, 255:red, 0; green, 0; blue, 0 }  ][line width=0.75]    (10.93,-4.9) .. controls (6.95,-2.3) and (3.31,-0.67) .. (0,0) .. controls (3.31,0.67) and (6.95,2.3) .. (10.93,4.9)   ;
%Curve Lines [id:da14799908671254725] 
\draw    (353.5,207.81) .. controls (344,196.63) and (326,150.63) .. (331.5,139.81) ;
\draw [shift={(337.33,174.95)}, rotate = 249.44] [color={rgb, 255:red, 0; green, 0; blue, 0 }  ][line width=0.75]    (10.93,-3.29) .. controls (6.95,-1.4) and (3.31,-0.3) .. (0,0) .. controls (3.31,0.3) and (6.95,1.4) .. (10.93,3.29)   ;
%Curve Lines [id:da6213471243029419] 
\draw    (156,84) .. controls (177.5,107.81) and (267.5,138.81) .. (331.5,139.81) ;
\draw [shift={(239.97,124.24)}, rotate = 196.82999999999998] [color={rgb, 255:red, 0; green, 0; blue, 0 }  ][line width=0.75]    (10.93,-4.9) .. controls (6.95,-2.3) and (3.31,-0.67) .. (0,0) .. controls (3.31,0.67) and (6.95,2.3) .. (10.93,4.9)   ;

% Text Node
\draw (180,155) node [anchor=north west][inner sep=0.75pt]    {$a$};
% Text Node
\draw (355.5,210.81) node [anchor=north west][inner sep=0.75pt]    {$d$};
% Text Node
\draw (159,65) node [anchor=north west][inner sep=0.75pt]    {$b$};
% Text Node
\draw (337,119) node [anchor=north west][inner sep=0.75pt]    {$c$};
% Text Node
\draw (228,269) node [anchor=north west][inner sep=0.75pt]    {$V$};
% Text Node
\draw (70,148) node [anchor=north west][inner sep=0.75pt]    {$P$};
% Text Node
\draw (224,76) node [anchor=north west][inner sep=0.75pt]    {$Iso.\ exp^{n}$};
% Text Node
\draw (350,152) node [anchor=north west][inner sep=0.75pt]    {$Adi.\ exp^{n}$};
% Text Node
\draw (221,202) node [anchor=north west][inner sep=0.75pt]    {$Iso.\ comp^{n}$};
% Text Node
\draw (92,142) node [anchor=north west][inner sep=0.75pt]    {$Adi.\ comp^{n}$};
% Text Node
\draw (298,114) node [anchor=north west][inner sep=0.75pt]    {$T_{H}$};
% Text Node
\draw (358,192) node [anchor=north west][inner sep=0.75pt]    {$T_{C}$};


\end{tikzpicture}
	\begin{itemize}
	\item \underline{Carnot's Theorem}: For two given heat reserviors, no heat engine can have a thermal efficiency higher than a carnot engine.
	\item \underline{ Corollary}: All carnot engines operating between the same temp. have the same $\eta$
	\end{itemize}
	\item \underline{Thermodynamic Temperature Scaling}\\
	\[ \eta=1-\frac{Q_C}{Q_H}\]
	For carnot engine $\displaystyle \eta=1-\frac{T_C}{T_H}\\
	|Q_H|=RT_Hln\left(\frac{V_c}{V_b}\right)\\
	|Q_C|=RT_Cln\left(\frac{V_d}{V_a}\right)$
	\item \underline{Entropy}
	\begin{itemize}
	\item $\displaystyle \frac{\partial Q}{\partial T}$ is a state function.
	\item $\displaystyle \partial S^t=\frac{\partial Q_{rev}}{T} \ \ \ \ \ 
	\partial S \geq \frac{\partial Q}{T}$\\
	Equality holds when process is reversible.
	\end{itemize}
	\item \underline{Entropy change for an ideal gas}\\
	$\displaystyle \partial U=\partial Q_{rev}-PdV\\
	\partial H= \partial Q_{rev}+V\partial P$\\
	For Ideal gas, $\displaystyle \partial H= C_p^{ig}dT\ , V=\frac{RT}{P}$\\
	So, $\displaystyle \frac{\partial Q_{rev}}{T} = \frac{\partial H}{T} - \frac{R\partial P}{P} \\ \\
	\implies \partial S = C_P^{ig}\frac{\partial T}{T} - \frac{R\partial P}{P}$
	So, $\displaystyle \frac{\Delta S}{R} = \int_{T_0}^{T}\frac{C_P}{RT}dT-ln\left(\frac{P}{P_0}\right)$\\
	By this we can prove for adiabatic equation with constant $C_P$\\
	$TP^{\frac{1-\gamma}{\gamma}}=constant$\ \ \ \ \ \ \ \ $\gamma=\frac{C_P}{C_V}$\
	\item \underline{Entropy balance for open system}\\
	\[\frac{\partial}{\partial t}[mS]_{controlled\ vol.}= -\Delta(\dot m S)+ \sum_j\frac{\dot Q_j}{T_j}+\dot S_G\]
	\item\underline{Calculation of ideal work}
	\[ \Delta(\dot m S)_{flow\ surface}-\sum_j\frac{Q_j}{T_{\sigma,j}}=\dot S_G\geq0\]
	\begin{itemize}
		\item SS flow process req, work\\
		$W_{ideal} \rightarrow$ minimum amount of work required to obtain a change of state
		\item SS flow producing work\\
		$W_ideal \rightarrow$ maximum amount of work produced for a given change
		\item SS rev process\\
		\[ (\Delta(\dot mS))_{fs}=\frac{\dot Q}{T_\sigma}\implies \dot Q=T_\sigma\Delta(\dot mS)_{fs}\]
		So, $\displaystyle \Delta\left[\dot m\left(H+\frac{1}{2}u^2+gz \right) \right] = T_\sigma\Delta(\dot mS)_{fs}+\dot W_{S\ (rev)}$\\
		$W_{ideal}\implies \Delta(\dot mH)-T_\sigma\Delta(\dot mS)_{fs}$ \ \ \ \ (mostly)\\
		For single inlet, outlet $\dot m_1 = \dot m_2 = \dot m_3$\\
		$\therefore W_{ideal} = m(\Delta H-T\Delta S)\\
		\displaystyle \dot W_{ideal}= \Delta\left[\dot m(H+\frac{1}{2}u^2+gz\right]-\dot Q\\
		\dot W_{lost}=\dot W_S-\dot W_{ideal}=T_\sigma\dot S_G\geq 0$ By putting original $Q$ in $\dot W_S$
	\end{itemize}
	\item \underline{Closed System of 'n' moles}
	\begin{itemize}
		\item $U\equiv Q+W$
		\item $H\equiv U+PV$
		\item \underline{Free energy}
		\item $G\equiv H-TS$
		\item $A\equiv U-TS$ (Helmholtz free energy)
	\end{itemize}
	These lead to,
	\begin{itemize}
		\item $\displaystyle d(nH)=Td(nS)+(nV)dP\\
		\implies \left(\frac{\partial T}{\partial P}\right)_S=\left(\frac{\partial V}{\partial S}\right)_P$\\
		\item $\displaystyle d(nU)=Td(nS)-Pd(nV)\\
		\implies \left(\frac{\partial T}{\partial V}\right)_S=-\left(\frac{\partial P}{\partial S}\right)_V$
		\item $\displaystyle dG=(nV)dP-(nS)dT\\
		\implies \left(\frac{\partial V}{\partial T}\right)_P=\left(\frac{\partial S}{\partial P}\right)_T$
		\item $\displaystyle dA=-Pd(nV)-(nS)dT\\
		\implies \left(\frac{\partial P}{\partial T}\right)_V=\left(\frac{\partial S}{\partial V}\right)_T$
	\end{itemize}
	\item \underline{Enthalpy \& Entropy as function of T \& P}
	\item $\displaystyle \partial H(T,P)=\left(\frac{\partial H}{\partial T}\right)_PdT+ \left(\frac{\partial H}{\partial P}\right)_TdP$\\
	1. Writing $\partial H$ at const. $T$ \& $P$\\
	2. Using Maxwell's equations\\
	$\displaystyle \partial H(T,P)=C_PdT+\left[V-T\left(\frac{\partial V}{\partial T}\right)_P\right]dP$
	\item $\displaystyle \partial S(T,P)=\left(\frac{\partial S}{\partial T}\right)_PdT+\left(\frac{\partial S}{\partial P}\right)_TdP\\
	\partial S(T,P)= \frac{C_P}{T}\partial T-\left(\frac{\partial V}{\partial T}\right)_P\partial P$\\ \\
	\underline{Alternative form for liquid}\\
	$\displaystyle \left(\frac{\partial S}{\partial P}\right)_T=-\left(\frac{\partial V}{\partial T}\right)=-V\beta\\
	\left(\frac{\partial H}{\partial P}\right)_T=(1-\beta T)V$\\
	So,\\
	$\partial H=C_PdT+(1-\beta V)VdP\\
	\partial S=\frac{C_P}{T}\partial T-\beta VdP$\\
	For liquids, $\beta$ \& $V$ are weak functions of $P$
	\item \underline{Internal Energy \& Entropy as a function of $T$, $V$}\\
	$\displaystyle \partial U= C_V\partial T+\left(\frac{\beta T}{\kappa}-P\right)\partial V\\
	\partial S=\frac{C_V}{T}\partial T+\frac{\beta}{\kappa}\partial V$
	\item \underline{Gibbs' Free Energy as a generating function}\\
	$\displaystyle \partial G=V\partial P-S\partial T\\
	\partial\left(\frac{G}{RT}\right)=\frac{V}{RT}\partial P- \frac{H}{RT^2}\partial T$\\ 
	$G$ and $T$ are both variables so partial differentiation.
	So,\\
	$\displaystyle \frac{V}{RT}=\left(\frac{\partial\left(\frac{G}{RT}\right)}{\partial P}\right)_T$ \ \ \ and \ \ \ $\displaystyle \frac{H}{RT}=-\left[T\frac{\partial\left(\frac{G}{RT}\right)}{\partial T}\right]_P$
	\item \underline{Residual Properties}\\
	$M^R=M-M^{ig}$ where $M$ \& $M^{ig}$ are at the same temperature and pressure\\
	$V^R=V-V^{ig} = \displaystyle V=\frac{RT}{P}$ \ \ \ \ \ $^R=$ Residual\\
	So, $G^R=G-G^{ig}$\\
	$\displaystyle\left[\frac{\partial\left(\frac{G}{RT}\right)}{\partial P}\right]_T=\frac{V^R}{RT}\ \ \ ; \ \ \left[\frac{\partial\left(\frac{G^R}{RT}\right)}{\partial T}\right]_P=-\frac{H^R}{RT^2}\\
	\partial\left(\frac{G^R}{RT}\right)=\frac{V^R}{RT}\partial P-\frac{H^R}{RT^2}\partial T$\\
	\underline{At constant Temperature}\\
	\[ \partial\left(\frac{G^R}{RT}\right)=\frac{V^R}{RT}\partial P\\
	\frac{G^R}{RT}=\int^P_0\frac{V^R}{RT}\partial P +\left(\frac{G^R}{RT}\right)_P\ \ ; \ \ \left(\frac{G^R}{RT}\right)_P=0\\\]
	Using $\displaystyle  V^R=V-\frac{RT}{P}=\frac{RT}{P}(Z-1)\ \ \ \because \frac{G^R}{RT}=\int_0^P\frac{\partial P}{P}(Z-1)$\\
	$G=G^{ig}+G^R$\\
	So, $\displaystyle \frac{H^R}{RT}=-T\int_0^P\left[\left(\frac{\partial Z}{\partial T}\right)_P\frac{\partial P}{P}\right]\ \ \&\ \ \frac{S^R}{R}=\frac{H^R}{RT}-\frac{G^R}{RT}$\\
	\[\implies\displaystyle \frac{S^R}{R}=-T\int_0^P\left(\frac{\partial Z}{\partial T}\right)_P\frac{\partial P}{P}-\int^P_0\frac{(Z-1)\partial P}{P}\]
	By equation of state $\displaystyle Z=1+B'P=1+\frac{BP}{RT}$\\
	$\displaystyle \frac{G^R}{RT}=\frac{BP}{RT} \ \ , \ \ \frac{H^R}{RT}=\frac{P}{R}\left[\frac{B}{T}-\frac{\partial B}{\partial T}\right]\\
	\frac{S^R}{R}=-\frac{R}{R}\frac{\partial B}{\partial T}$\\\\
	\underline{Pressure Explicit Form}\\
	$\displaystyle P=\rho ZRT\\
	\partial P=RT[Z\partial\rho+\rho\partial Z]\\
	\frac{\partial P}{P}=\frac{\partial\rho}{\rho}+\frac{\partial Z}{Z}$
	Now, we know that
	$\displaystyle \frac{H^R}{RT}=-T\int^P_0\frac{\partial Z}{\partial T}$\\
	and 
	$\displaystyle \partial\left(\frac{G}{RT}\right)=\frac{V}{RT}\partial P-\frac{H}{RT}\partial T$\\
	Substituting $\displaystyle \frac{\partial P}{P}$ with these\\\\
	$\displaystyle \frac{G^R}{RT}=\int^P_0\frac{(Z-1)\partial\rho}{\rho}+Z-1-lnZ\\
	\frac{H^R}{RT}=-T\int^P_0\left(\frac{\partial Z}{\partial T}\right)_\rho\frac{\partial \rho}{\rho}+Z-1\\
	\frac{S^R}{R}=lnZ-T\int^P_0\left(\frac{\partial Z}{\partial T}\right)_P\frac{\partial \rho}{\rho}-\int^P_0(Z-1)\frac{\partial\rho}{\rho}$
	\item\underline{Clayperon Equation}\
	\[\left.\frac{\partial P}{\partial T}\right\rvert_{along\ a\ coexistence\ curve}=\frac{\Delta H_{tans}}{T\Delta V_{trans}}\]\\ \\ \\
	\underline{Proof}\\
	Taking Gibbs' Free energy as a function of Temperature and Pressure (as it is the only one which has a continuous graph at transition against temperature\\
	$\displaystyle G(T,P)\\
	G^t=m_lG_l+m_vG_b\\
	\partial G^t=\partial m_lG_l+\partial m_vG_v\\
	0=\partial m_lG_l-\partial m_vG_v\\$
	as $\partial G=V\partial P-S\partial T$ as $\partial P\ \&\ \partial T$ are zero at transition\\
	$\therefore G^l=G^v$\\
	

\tikzset{every picture/.style={line width=0.75pt}} %set default line width to 0.75pt        

\begin{tikzpicture}[x=0.75pt,y=0.75pt,yscale=-1,xscale=1]
%uncomment if require: \path (0,300); %set diagram left start at 0, and has height of 300

%Shape: Axis 2D [id:dp28279207813556284] 
\draw  (168,236.53) -- (461.5,236.53)(197.35,36) -- (197.35,258.81) (454.5,231.53) -- (461.5,236.53) -- (454.5,241.53) (192.35,43) -- (197.35,36) -- (202.35,43)  ;
%Curve Lines [id:da6696585544486019] 
\draw  [dash pattern={on 4.5pt off 4.5pt}]  (258,81) .. controls (283.5,117.81) and (339.5,152.81) .. (384.5,164.81) ;
%Curve Lines [id:da7300380853997266] 
\draw    (254,93) .. controls (285.5,127.81) and (336.5,144.81) .. (384.5,144.81) ;
%Straight Lines [id:da7474660884062012] 
\draw  [dash pattern={on 0.84pt off 2.51pt}]  (310.5,131.81) -- (308.5,234.81) ;

% Text Node
\draw (122,139) node [anchor=north west][inner sep=0.75pt]    {$G$};
% Text Node
\draw (299,238) node [anchor=north west][inner sep=0.75pt]    {$T^{t}$};
% Text Node
\draw (452,246) node [anchor=north west][inner sep=0.75pt]    {$T$};
% Text Node
\draw (386,155) node [anchor=north west][inner sep=0.75pt]    {$v$};
% Text Node
\draw (386,132) node [anchor=north west][inner sep=0.75pt]    {$l$};


\end{tikzpicture}\\
	For a  pure species co-existing in 2 phases $(\alpha,\beta)$ in equation\\
	\[G^\alpha=G^\beta\]\ \ \ \ [molar/specific Gibbs' free energy]\\
	And we can also say\\
	$ \displaystyle\partial G^\alpha=\partial G^\beta \\
	V^\alpha\partial p^{sat}-S^\alpha\partial T^{sat}=V^\beta\partial p^{sat}-S^\beta\partial T^{sat} \\
	\implies \frac{\partial p^{sat}}{\partial T^{sat}}=\frac{(S^\beta-S^\alpha)}{\Delta V_{transition}}\\
	dH=TdS+VdP\ \ \ \ \ \ \ \ \ \ [dP=0]\\
	\Delta H^{\alpha\beta}=T\Delta S^{\alpha\beta}\\
	\implies \frac{\partial p^{sat}}{\partial T^{sat}}=\frac{\Delta H^{\alpha\beta}}{T\Delta V^{\alpha\beta}}$
	
	
	\end{itemize}
\end{document}